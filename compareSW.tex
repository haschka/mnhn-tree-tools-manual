\section{compareSW}

\subsection{General}

The \emph{compareSW} tool calculates the mean distance and standard deviation
between two datasets using the Smith Waterman distance. This tool only
works with nucleotide sequences.

\subsection{Usage}

One can call the \emph{compareSW} tool like:
\lstset{language=bash,
  caption={Calling the \emph{compareSW} tool},
  label=lst-compareSW-call}
\begin{lstlisting}
compareSW [fasta1] [fasta2] [n-threads]
\end{lstlisting}
with the following arguments:
\begin{enumerate}
\item \emph{fasta1}: A FASTA file containing a set of sequences.
\item \emph{fasta2}: An other FASTA file containing a set of sequences.
\item \emph{n-threads}: The number of threads that this computation might use.
\end{enumerate}

\subsection{Algorithm}

The tool calculates the mean and the standard deviation of the distances
between the sequences in the two datasets. If the first dataset has
$n$ sequences and the second $j$, the number of total distances
calculated is hence, $N=nj$. The Smith Waterman algorithm is the same
as outlined in section \ref{sec-dbscan-algorithm}. 

\subsection{Example}
\lstset{language=bash,
  caption={Example of the \emph{compareSW} tool},
  label=lst-compareSW-example}
\begin{lstlisting}
compareSW test1.fasta test2.fasta 8
\end{lstlisting}
This example computes the mean distance between the sequences in test1.fasta
and test2.fasta using 8 cores. 

\subsection{Implementation}
The functions to calculate the mean and standard deviation are
implemented in \emph{comparison.c}. The Interface is implemented in
\emph{compareSW.c}.
