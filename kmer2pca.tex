\section{kmer2pca} \label{sec-kmer2pca}

\subsection{General}

The \emph{kmer2pca} tool implements the calculation of and projection onto
principal components. The tools yields the projections onto the
principal components as well as the eigenvalues of covariance matrix.

\subsection{usage}

\lstset{language=bash,
  caption={Calling the \emph{kmer2pca} tool},
  label=lst-kmer2pca-call}
\begin{lstlisting}
kmer2pca [kmer-file] [projections] [eigenvalues] [dimensions] [n-threads]
\end{lstlisting}
The \emph{kmer2pca} tool has the following arguments.
\begin{enumerate}
  \item \emph{kmer-file} A file containing k-mer frequencies as
    calculated by the fasta2kmer tool presented in section
    \ref{sec-fasta2kmer}
  \item \emph{projections} The file projections onto the
    principal components are written to. This file contains
    tabulator separated values, each line corresponds to the line in
    k-mer frequencies input file.
  \item \emph{eigenvalues} The file the eigenvalues of
    covariance matrix are written to. The eigenvalues are sorted from
    smallest to largest.
  \item \emph{dimensions} The number of principal components to
    project on and as such the number of dimensions of the resulting
    subspace. The k-mers are projected onto the [dimension]
    principal components corresponding to [dimension] highest
    eigenvalues.
  \item \emph{n-threads} The number of threads to be made available
    for this computation. A good choice is the number of logical cores
    available on the machine.
\end{enumerate}

\subsection{example}

\lstset{language=bash,
  caption={Example call \emph{kmer2pca} tool},
  label=lst-kmer2pca-example}
\begin{lstlisting}
fasta2kmer test.kmer test.pca test.ev 7 8
\end{lstlisting}
In this example we calculate the projections into a seven dimensional
subspace spun by the 7 principal components with the highest
variance. The projections  onto these
7 principal components of the k-mers stored in test.kmer are written
to test.pca. The eigenvalues of the 
covariance matrix built from the kmers in test.kmer are written to the
test.ev file. 8 threads are used to build the covariance
matrix.

\subsection{algorithm}

The PCA is straight forward implemented in its most common
form. Having the vectors of k-mer frequencies for each sequence at
hand we can call each k-mer a feature, of which we have the number of sequences
samples. In order perform PCA we first need to obtain a feature
covariance matrix, and hence calculate the covariance between each pair of
k-mers in the dataset. We first calculate the mean of all
k-mers, let us call a single k-mer $X$:
\begin{equation}
  \bar{X} = \frac{1}{n}\sum_{i=1}^{n}X_i,
\end{equation}
where $n$ is the number of sequences. For two different k-mers, let us
call them $X$ and $Y$ we can than calculate the covariance:
\begin{equation}
  \sigma_X\sigma_Y =
  \frac{1}{n}\sum_{i=1}^{n}(X_i-\bar{X})(Y_i-\bar{Y}). \label{eqn-covariance}
\end{equation}
and further built the covariance matrix for all available covariances
from all available k-mers.
\begin{equation}
  C = \left[
    \begin{array}{ccc}
      \sigma_1\sigma_1 & \cdots & \sigma_1\sigma_n \\
      \vdots & \ddots & \vdots \\
      \sigma_n\sigma_1 & \cdots & \sigma_n\sigma_n
    \end{array}
    \right]. \label{eqn-covariance-matrix}
\end{equation}
From this covariance matrix solve the eigenvalue equation:
\begin{equation}
  Cv_i=\lambda_i v_i \label{eqn-eigen},
\end{equation}
where $v_i$ are the eigenvectors or in the sense of PCA the principal
components while $\lambda_i$ are the associated eigenvalues.

Once the principal components are obtained they are sorted together
with the associated eigenvalues. In order to calculate the projections
a sample $S$ and hence a sequence in its k-mer representation has to be
projected onto the principal components $v_i$ with the highest
eigenvalues. The number of projections is given by the
\emph{dimension} parameter of the tool. The projections $p_i$ are the simple
inner product of a k-mer representation of a sequence $S$ with the
eigenvector $v_i$,
\begin{equation}
  p_i = <v_i|S>.
\end{equation}
As such we have effectively diminished the dimensionalty to the number of
\emph{dimensions} as selected.

The algorithm is implemented by parallization on different
levels. The algorithm spends for a great number of samples most of its
time calculating the matrix elements of equation
(\ref{eqn-covariance}). The matrix itself rests in most cases
of a small rank, the number of individual k-mers. As such solving the
eigenvalues and eigenvectors ( principal components ) in equation
(\ref{eqn-eigen}) is especially for k-mers of small $k$ fast. Hence, we
focused our efforts on the calculation of the matrix elements in
(\ref{eqn-covariance}). As the covariance matrix is symmetric we
calculate only the half of the elements transposing them into the
other half. Each row of the matrix is calculated in its own thread,
allowing us to calculate several rows on several cores in
parallel. Besides this we used Single Instruction Multiple Data (SIMD)
instructions in evaluating equation (\ref{eqn-covariance}). We do this
in calculating partial sums:
\begin{eqnarray}
  \sigma_X
  \sigma_Y[1]&=&\sum_{i=1}^{n/4}(X_{[(i+4)n/4]}-\bar{X})(Y_{[(i+4)n/4]}-\bar{Y}) \\
  \sigma_X
  \sigma_Y[2]&=&\sum_{i=2}^{n/4}(X_{[(i+4)n/4]}-\bar{X})(Y_{[(i+4)n/4]}-\bar{Y}) \\
  \sigma_X
  \sigma_Y[3]&=&\sum_{i=3}^{n/4}(X_{[(i+4)n/4]}-\bar{X})(Y_{[(i+4)n/4]}-\bar{Y}) \\
  \sigma_X
  \sigma_Y[4]&=&\sum_{i=4}^{n/4}(X_{[(i+4)n/4]}-\bar{X})(Y_{[(i+4)n/4]}-\bar{Y}) \\
  \sigma_X
  \sigma_Y &=& \sum_{i=1}^4\sigma_X\sigma_Y[i].
\end{eqnarray}
Further the partial sums are implemented by the Kahan summation
algorithm \cite{kahan}, avoiding floating point errors in summing a
large number of values.

Calculating the eigenvectors and eigenvalues is performed by calling
the optimized \emph{dsyev} routine from the LAPACK libraries \cite{lapack}.

\subsection{Implementation}

Both the algorithm and the interface are implemented in \emph{kmer2pca.c}.
