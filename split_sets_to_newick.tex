\section{split\_sets\_to\_newick} \label{sec-ssnewick}

This tool allows you to create a Newick \cite{newick} tree file from the
\emph{split-set} and \emph{output} files that have been generated by
an adaptive clustering run as outlined in section
\ref{sec-adaptive-clust}.

\subsection{Usage}

The tool can be used as outlined in the following:

\lstset{language=bash,
  caption={Calling the \emph{split\_sets\_to\_newick} tool},
  label=lst-ssnewick-call}
\begin{lstlisting}
split_sets_to_newick [filter] [lengths-from-output] [dimensions] \
  [outfile] [minPoints] [file1 ... filen] > [newick]
\end{lstlisting}
with the arguments:
\begin{enumerate}
\item \emph{filter} One may use 0, for no filter or any other number
  to define the
    minimum cluster size to occur in the resulting tree. Filtering the
    tree in this way changes the labels on the tree. In order to retain correct
    labels we suggest to use the
    \emph{filter\_split\_sets\_by\_min} tool, shown in section
    \ref{sec-filter-split}, prior using this tool and to
    apply no filter at this step. This parameter is kept for convenience.
  \item \emph{lengths-from-output} Allows us to define
    distances between nodes of the
    Newick tree utilizing the parameters found in the output file of an
    adaptive clustering run performed with one of the tools shown in section
    \ref{sec-adaptive-clust}. If you do not choose to tree lengths
    proportional to density differences between layers choose 0 here,
    1 if your run uses an $L_1$ based distance calculation and 2 if your
    run uses an $L_2$ based distance calculation. 1 uses a hypercube
    with a sidelength of $2\epsilon$ to calculated the volume, 2 uses
    a hypersphere with radius $\epsilon$.
  \item \emph{dimensions} As the lengths between the nodes correspond
    to density differences we need the dimensions in order to be able
    to calculate the volume using the different $\epsilon$ radiuses found in the
    \emph{outfile} file. As a basic guidance choose the number of
    projections onto principal components that you have retained in
    case this run was performed with PCA data. Otherwise choose 1 or a value to
    your convenience.
  \item \emph{outfile} The \emph{output} file from an adaptive
    clustering run generated from one of the tools outlined in section
    \ref{sec-filter-split}.
  \item \emph{minPoints} The \emph{minPoints} argument used during the
    adaptive clustering run which is needed in order to correctly calculate the
    correct density differences.
  \item \emph{file1$\ldots$filen} All the \emph{split-set} files
    to use to built this tree from. In order to select them in general
    wildcards can used on shells like the bash by i.e. using something
    like \begin{verbatim}/path/prefix*\end{verbatim}.
  \item \emph{newick} The created Newick tree file. 
\end{enumerate}
The arguments \emph{dimensions}, \emph{outfile} and \emph{minPoints}
are to be omitted if 0 was chosen for the \emph{lengths-from-output}
argument.

\subsection{Algorithm}
If the lengths in the newick file are chosen to be density
differences, the density is calculated from the $\epsilon$ value in
the output file of an adaptive clustering run.
The formulas used for the volume are:
\begin{equation}
  V = (2\epsilon)^d,
\end{equation}
for the hypercubic $L_1$ distance case, and
\begin{equation}
  V = \epsilon^{d}\frac{\pi^{d/2}}{\Gamma(d/2+1)}
\end{equation}
for the hyperspherical case. Here $d$ is the number of dimensions
and $\Gamma$ is the $\Gamma$-function.

The algorithm builds the Newick file using a recursive function traversing
the nodes. Connections between nodes in different layers are made
if from a layer corresponding to denser clusters
a cluster is contained at least at 80\% in a layer corresponding to
more diffuse clusters. 

\subsection{Example}

\lstset{language=bash,
  caption={Example of the \emph{split\_sets\_to\_newick} tool},
  label=lst-ssnewick-example}
\begin{lstlisting}
  split_sets_to_newick 0 2 7 outfile 4 clusters/out-* > tree.dnd
\end{lstlisting}
Here we generate a Newick tree that we store in the \emph{tree.dnd} file.
The lengths of the tree are density difference based, from an adaptive
clustering run whose output has been stored in \emph{outfile} and
whose \emph{split-sets} have been stored to \emph{clusters/out-X}
files. A \emph{minPoints} argument of 4 was used during the adaptive
clustering run, which was PCA based and operated on a seven
dimensional subspace. 

\subsection{Implementation}
The interface for Newick tree generation is implemented in
\emph{split\_sets\_to\_newick.c}. The algorithms that allow us to
generate a tree are found in \emph{cluster\_io.c}.
