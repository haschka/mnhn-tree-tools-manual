\section{split\_set\_to\_matrix\_X}

This are two tools to generate a matrix view of a tree that can then be
plotted with the \emph{gnuplot} \cite{gnuplot} program. The matrix
view basically, highlights each sequence as a point on a line,
coloring sequences belonging to the same cluster in the same
color. This is useful for instance in the verification of simulated or
real datasets where the sequences are already ordered by clusters and
hence should generate lines in the same color. This also shows as one
stacks these lines on top of each other how clusters merge as the
$\epsilon$ value changes.

\subsection{Usage}

The tools \emph{split\_set\_to\_matrix\_annotation} and
\emph{split\_set\_to\_matrix\_line} can be used together to form a
readable output file. 

\lstset{language=bash,
  caption={Calling the \emph{split\_set\_to\_matrix} tools},
  label=lst-ssmatrix-call}
\begin{lstlisting}
split_set_to_matrix_annotation [fasta] >> [matrix]
split_set_to_matrix_line [fasta] [split-set] >> [matrix]
\end{lstlisting}
where:
\begin{enumerate}
  \item \emph{fasta} Is a FASTA file that one has generated clusters
    from for instance in an adaptive clustering, as the ones outlined
    in section \ref{sec-adaptive-clust}
  \item \emph{split-set} A binary file holding the clusters obtained
    from an adaptive clustering run.
  \item \emph{matrix} The matrix to store a line to.
\end{enumerate}
The \emph{split\_set\_to\_matrix\_annotation} call in itself is not
needed, it basically annotates each sequence with the internal
sequence number, or the number of sequence occurrence in the original
FASTA file, starting with 0 for the first sequence.

\subsection{Example}
One has run an adaptive clustering run that yielded 86 different
clustering layers (of a tree) and hence, 86 split set files in using
one of the adaptive clustering tools shown in section \ref{sec-adaptive-clust}
\lstset{language=bash,
  caption={Example usage of the \emph{split\_set\_to\_matrix} tools},
  label=lst-ssmatrix-example}
\begin{lstlisting}
for((i=0;i<85;i++))
do split_set_to_matrix_line test.fasta /tmp/out`printf %04d $i` >> test.matrix
done
\end{lstlisting}
In this example the 86 \emph{split-sets} for each layer of the
adaptive clustering run are successively as lines written to the
\emph{test.matrix} file.
The \emph{test.matrix} file can then be visualized by \emph{gnuplot}.
On the \emph{gnuplot} console one might issue the command:
\lstset{language={},
  caption={Using gnuplot to plot a matrix from \emph{split\_set\_to\_matrix} tools},
  label=lst-ssmatrix-example-plot}
\begin{lstlisting}
plot 'test.matrix' matrix with image
\end{lstlisting}
to obtain a map with different colors for each cluster. 
  
