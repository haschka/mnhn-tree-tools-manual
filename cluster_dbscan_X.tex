\section{cluster\_dbscan\_X} \label{sec-dbscan-cluster}

\subsection{General}

The \emph{cluster\_dbscan\_PCA}, \emph{cluster\_dbscan\_kmerL1},
\emph{cluster\_dbscan\_kmerL2}, \emph{cluster\_dbscan\_SW} and
\emph{cluster\_dbscan\_SW\_GPU} implement clustering of a dataset of
sequences using the DBSCAN \cite{dbscan} algorithm. The different
variants implement DBSCAN clustering using different sequence-sequence
distances and platforms. 

\subsection{Usage}

\lstset{language=bash,
  caption={Calling the \emph{cluster\_dbscan} tools},
  label=lst-dbscan-call}
\begin{lstlisting}
cluster_dbscan_pca [fasta] [pcaproj] [dimensions] [epsilon] [minPoints] \
   [outfiles-fasta-prefix] [outfiles-values-prefix]

cluster_dbscan_kmerL1 [fasta] [kmers] [epsilon] [minPoints] \
   [outfiles-fasta-prefix]

cluster_dbscan_kmerL2 [fasta] [kmers] [epsilon] [minPoints] \
   [outfiles-fasta-prefix]

cluster_dbscan_SW [fasta] [epsilon] [minPoints]
   [outfiles-fasta-prefix]

cluster_dbscan_SW_GPU [fasta] [epsilon] [minPoints] \
   [outfiles-fasta-prefix]   
\end{lstlisting}
Where the different tools perform as disguised by their name:
\begin{enumerate}
  \item \emph{cluster\_dbscan\_pca}: Clustering of sequences in a PCA
    subspace such as the one obtained using the \emph{kmer2pca} tool
    described in section \ref{sec-kmer2pca}.
  \item \emph{cluster\_dbscan\_kmerL1}: Clustering of sequences in a
    k-mer representation applying the $L_1$-norm also known as the
    Manhattan distance between the k-mer frequency vectors that
    represent each sequence as a distance between them.
  \item \emph{cluster\_dbscan\_kmerL2}: Clustering of sequences in a
    k-mer representation applying the $L_2$-norm also known as the
    Manhattan distance between the k-mer frequency vectors that
    represent each sequence as a distance between them.
  \item \emph{cluster\_dbscan\_SW}: Clustering of sequences using the
    Smith-Waterman distance measure.
  \item \emph{cluster\_dbscan\_SW\_GPU}: Clustering of sequences using
    the Smith-Waterman distance measure on an OpenCL supported device,
    for instance a GPU.
\end{enumerate}
The tools have the following arguments:
\begin{enumerate}
  \item \emph{fasta}: A FASTA file containing the sequences to be
    clustered. In the case of Smith-Waterman clustering the FASTA file
    has to container nucleic sequences as clustering of proteic
    sequences under a Smith-Waterman distance is not implemented. In
    case the Smith-Waterman distance is evaluated on GPUs the maximum
    sequence length is limited to 300 nucleotide bases.
  \item \emph{pcaproj}: The projections onto the principal components
    as calculated by the \emph{kmer2pca} tool. 
  \item \emph{dimensions}: The number of dimensions, principal
    components found in the file.
  \item \emph{epsilon}: The epsilon neighborhood radius as used by the DBSCAN
    \cite{dbscan} algorithm.
  \item \emph{minPoints}: The minimum number of sequences to be found
    within an epsilon neighborhood in order to form or expand a
    cluster according to the DBSCAN algorithm.
  \item \emph{outfiles-fasta-prefix}: The DBSCAN based tools all
    generate a FASTA file per cluster found. With this argument you
    can tell the program where to store those clusters. The number of
    clusters to be generated is capped at 500 clusters. The resulting
    files will be named according to your prefix with a number at the
    end for each different cluster.
  \item \emph{outfiles-values-prefix}: If this parameter is added to
    the PCA version of the DBSCAN implementation the clusters are not
    only stored as individual fasta files but also as files containing
    the projections onto the principal components for the clustering
    in question. This for instance allows one to draw the different
    clusters into visual representations of the PCA projections.
\end{enumerate}

\subsection{Algorithm} \label{sec-dbscan-algorithm}

The utility implements the DBSCAN algorithm straight forward as
described in its original article \cite{dbscan}. The DBSCAN algorithm
is perfect in finding \emph{density connected} subsets of a
dataset. The algorithm as such find all connected ensembles in a
dataset that exceed a certain density:
\begin{equation}
  \rho = \frac{n}{V(\epsilon)} \label{eqn-density}
\end{equation}
where V is the Volume within an epsilon neighborhood defined by the
radius $\epsilon$. $n$ corresponds to the number of sequences within
the aforementioned volume, and is defined using the \emph{minPoints}
parameter. The formula for the volume depends on the metric used, but
is hypercubic for $L_1$ measurements, and hyperspheric for $L_2$
measurements. Sequences scattered in sequence space with a density
below $\rho$ are found to be outliers and will not be taken into
account and are hence not to be found in the resulting clusters. 

The different distance measures in the \emph{region expand method} as
described in the original paper use various means of optimization and
parallization. In the case of PCA, kmerL1, kmerL2 distance methods up to eight
distances between sequences are calculated in parallel making use of
the AVX SIMD instructions. In the case of the Smith Waterman distance
measure an OpenCL (GPU) method is available allowing to efficiently calculate
multiple distances in parallel on OpenCL devices such as GPUs. 

The Smith Waterman algorithm calculates the Smith Waterman matrix and
defines the maximum value to be found within the matrix to be the
distance between two sequences. No implementation a Smith Waterman
algorithm using protein sequences has been implemented. The algorithm
builds the matrix using the recurrence relation:
\begin{eqnarray}
s_1&=&M(j-1,i-1)+F(A(i),B(j)), \\ \nonumber
s_2&=&M(j,i-1)-G, \\ \nonumber
s_3&=&M(j-1,i)-G, \\ \nonumber
M(i,j)&=&\mathrm{max}(s_1,s_2,s_3).
\end{eqnarray}
where $A$ and $B$ are two sequences of the dataset to be compared and
$A(i)$ is the i-th letter of sequence $A$ and $B(j)$ is the j-th
letter of sequence $B$.
\begin{equation}
  F(A(i),B(j)) = \left\{
    \begin{array}{l}
      A(i) = B(j) : 4 \\
      A(i) \neq B(j) : -4,
    \end{array} \right.
\end{equation}
is the comparison relation. $G$ is the gap penalty defined to be 3 and
$M(i,j)$ the Sith Waterman matrix yielding the distance:
\begin{equation}
  d(A,B) = \mathrm{max}(M(i,j));
\end{equation}
In the case of the GPU implementation only three subsequent rows of
the matrix are stored in memory evaluating the maximum on the fly
calculating row after row of the Smith Waterman Matrix. This shall
allow to use more local memory that is closer to the compute unit, if
not only local registers on the graphics card for the computation.

\subsection{Example}

\lstset{language=bash,
  caption={Example of the \emph{cluster\_dbscan\_PCA} tool},
  label=lst-dbscan-example}
\begin{lstlisting}
  cluster_dbscan_PCA test.fasta test.pca 7 0.02 4 /tmp/outf /tmp/outv
\end{lstlisting}
This example performs DBSCAN based clustering using $L_2$ distances in
a 7 dimensional subspace with subspace projections stored in the
test.pca file. The input values for the DBSCAN run are $\epsilon=0.02$
and minpoints $n=4$. The resulting clusters will be stored in /tmp/
with in FASTA format with a file name of outf$N$, and as projections
onto the principal components outv$N$, where $N$ is an individual
number for each clusters. 

\subsection{Implementation}

The DBSCAN algorithm for all variants is implemented in \emph{dbscan.c}
The Interface for the Smith-Waterman variants is implemented in
\emph{cluster\_dbscan\_SW.c}.
The Interface for k-mer distance based $L_1$ and $L_2$ variants is
implemented in \emph{cluster\_dbscan\_kmer.c}.
The Interface for the PCA variant is implemented in
\emph{cluster\_dbscan\_pca.c}.









    
