\section{fasta2kmer} \label{sec-fasta2kmer}

\subsection{General}

The \emph{fasta2kmer} tool implements the calculation of k-mer
frequencies for each sequence to be found in a FASTA \cite{fasta}
file. The tool can calculate both k-mer frequencies for nucleic and
protein sequences. k-mers are all possible sequences for k
letters. The frequencies tell how often each of these k-mers is found
in the entire sequence.

\subsection{Usage}

\lstset{language=bash,
  caption={Calling the \emph{fasta2kmer} tool},
  label=lst-fasta2kmer-call}
\begin{lstlisting}
fasta2kmer [fasta-file] [kmer-length] [number-of-threads] [protein?] > kmerbase
\end{lstlisting}
The \emph{fasta2kmer} tool has the following arguments:
\begin{enumerate}
  \item \emph{fasta-file} The FASTA file containing the sequences to
    calculate k-mer frequencies of.
  \item \emph{kmer-length} The length of the k-mers to calculate
    frequencies from.
  \item \emph{number-of-threads} The number of individual threads to
    be available for this computation. A good choice is the number of
    logical cores available on the machine in question.
  \item \emph{protein?} Set 0 for nucleic sequences, 1 for protein
    sequences.
  \item \emph{kmerbase} The output file to write kmer frequencies
    for each sequence to. The format of this file is tabulator
    separated. The first entry in each line is 
    \emph{sequence\_N} where N is the sequence number attributed
    counting from 0 to N by occurrence of sequences in the FASTA
    file. The following entries the line, purely numeric are the
    frequencies of the k-mer in questions.
\end{enumerate}

\subsection{Example}

\lstset{language=bash,
  caption={Example call \emph{fasta2kmer} tool},
  label=lst-fasta2kmer-example}
\begin{lstlisting}
fasta2kmer test.fasta 5 8 0 > test.kmer
\end{lstlisting}
In this case one calculates 5-mer frequencies for the file test.fasta
using 8 threads. test.fasta contains nucleic sequences. The output ist
stored in test.kmer.

\subsection{Algorithm}

The nucleic k-mers are calculated using a binary scheme. As the
nucleic letters contain only four members A, C, G, T they can be
represented by 00, 01, 10, 11 in binary form. To generate all possible
k-mers we generate a binary number of $2k$ in length containing only
ones. I.e. for $k=3$ we generate $111 111 = 63$ and count down to $0$
generating all 64 3-mers by comparing the binary representation with
the letter code.

The protein code uses a different scheme generate proteic k-mers.
A recursive function called $k$ times shifting position counting to
from 0 to 19 is called generating all possible protein k-mers for the
length $k$.

The frequencies are calculated by comparing the letters at each overlay
position of the sequence to calculate frequencies for. An early stop
condition for a mismatching letter is implemented. The comparison is
performed in multithreaded fashion distributing the number sequences to
calcualte k-mer frequencies from across the number of threads
requested.

\subsection{Implementation}

The algorithms for the tool are implemented in \emph{kmers.c}
The interface to the algorithms is implemented in \emph{fasta2kmer.c}
