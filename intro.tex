\chapter{Introduction}

This manual is written as a guide for scientist using and continuing to
develop \emph{MNHN-Tools}. As such this guide provides you with a
installation procedure, detailed usage instructions and technical
details for every tool contained within this suite and a tutorial to get
the reader started with a typical clustering and density distance
based phylogenetics as well as tree building.

\section{License}
Copyright (C) 2019-2020 Thomas Haschka \newline
This software is provided 'as-is', without any express or implied
warranty.  In no event will the authors be held liable for any damages
arising from the use of this software.

Permission is granted to anyone to use this software for any purpose,
including commercial applications, and to alter it and redistribute it
freely, subject to the following restrictions:

\begin{enumerate}
  \item The origin of this software must not be misrepresented; you must not
    claim that you wrote the original software. If you use this software
   in a product, an acknowledgment in the product documentation would be
   appreciated but is not required.
  \item Altered source versions must be plainly marked as such, and must not be
    misrepresented as being the original software.
  \item This notice may not be removed or altered from any source,
    binary distribution and this manual.
\end{enumerate}

\section{General Overview}

\emph{MNHN-Tools} was developed in order to cluster and infer
phylogenetic evolution into repeated sequences in a single genome,
namely: α-satellites in primate genomes. During the development
the versatility of the tools presented herein was both discovered and
enforced. As such this tool is applicable to all kinds of data
discovery tasks on datasets of sequences. Notebly this tool has been
used for experiances around Deoxyribonucleic Acid (DNA) barcoding tasks
Operational Taxonomic Unit (OTU) detection.

\emph{MNHN-Tools} as such is a suite that allows you to gain insights
into a dataset composed of nucleic or protein sequences provided as
FASTA \cite{fasta} files. The tool allows you to cluster such a
dataset and to built phylogenetic trees based on a density- distance
based criteria. This is achieved by a repeated application of the
Density-Based Spatial Clustering of Applications with Noise (DBSCAN)
\cite{dbscan} algorithm. Clustering and gaining such sequence-sequence
dependency or phylogentic insights are at the heart of this
software. Nevertheless this software provides you with all kinds of
tools that might be of use, such as, but not limited to: The
generation of fasta datasets using simulated evolution, the projection
of FASTA files into a Principal Component Analyses (PCA) based subspace, the
evaluation of k-mer vectors from fasta datasets or the calulation of
clustering parameters such as the Silhouette \cite{silhoutte}
index on a dataset.
