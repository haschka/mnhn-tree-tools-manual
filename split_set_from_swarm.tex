\section{split\_set\_from\_swarm} \label{sec-ssswarm}

\subsection{General}

This tool generates a binary \emph{split-set} file describing
clusters in \emph{MNHN-Tree-Tools} format from a clustering obtained
with Swarm version 3 \cite{swarm2}.

\subsection{Usage}
The tool can be used in the following manner:
\lstset{language=bash,
  caption={Calling the \emph{split\_set\_from\_swarm} tool},
  label=lst-ssswarm-call}
\begin{lstlisting}
split_set_from_swarm [fasta] [swarm] [split-set]
\end{lstlisting}
with the following arguments:
\begin{enumerate}
  \item \emph{fasta} The original FASTA file that has been clustered with Swarm
    version 3 \cite{swarm2}. This is not the downsampled file that is
    used to perform clustering with the swarm tool, but the original dataset.
    (c.f. example).
  \item \emph{swarm} The clustering performed by swarm. Swarms output
    file
  \item \emph{split-set} The binary file containing the clustering to
    be created.
\end{enumerate}

\subsection{Example}
We imagine we would like to generate a clustering of
several sequences with swarm. Hence, we first have to eliminate duplicates
and format sequence multiplicity in a way that swarm understands it. The
\emph{sequence\_mulitplicity} tool, outlined in section
\ref{sec-seqmult}, is perfect for this task. As such we prepare the
swarm input file as follows:
\lstset{language=bash,
  caption={Generating a valable input file for swarm using the \emph{sequence\_multiplicity} tool},
  label=lst-sequencemultiplicity-example-swarm}
\begin{lstlisting}
sequence_multiplicity test.fasta > swarm-in.fasta
\end{lstlisting}
We than run the swarm tool to cluster our dataset.
\lstset{language=bash,
  caption={Calling Swarm before the \emph{split\_set\_from\_swarm} tool},
  label=lst-ssswarm-swarm-example}
\begin{lstlisting}
swarm swarm-in.fasta > swarm-output
\end{lstlisting}
and can finally built a \emph{split-set} cluster file that uses the interal
structures of MNHN-Tree-Tools.
\lstset{language=bash,
  caption={Example of the \emph{split\_set\_from\_swarm} tool},
  label=lst-ssswarm-example}
\begin{lstlisting}
split_set_from_swarm test.fasta swarm-output /tmp/split-set
\end{lstlisting}
Note that \emph{test.fasta} is used in the final call and not
\emph{swarm-in.fasta}. The tool can by tracking the correct
annotations find the correct sequences in the original dataset and
built a binary \emph{split-set} file accordingly.

\subsection{implementation}
This tool is implemented in \emph{split\_set\_from\_swarm.c}.
