\section{sequence\_multiplicity} \label{sec-seqmult}

\subsection{General}

The \emph{sequence\_multiplicity} finds unique sequences in a FASTA file and
allows one to know how often a unique sequence appears in the FASTA
file.

\subsection{Usage}

One can call the \emph{sequence\_multiplicity} tool like:
\lstset{language=bash,
  caption={Calling the \emph{sequence\_multiplicity} tool},
  label=lst-sequencemultiplicity-call}
\begin{lstlisting}
sequence_multiplicity [fasta] > [fasta-out]
\end{lstlisting}
with the arguments:
\begin{enumerate}
\item \emph{fasta} a FASTA file to find unique sequences in.
\item \emph{fasta-out} the unique sequences in FASTA format.
\end{enumerate}

\subsection{Example}
\lstset{language=bash,
  caption={Example of the \emph{sequence\_multiplicity} tool},
  label=lst-sequencemultiplicity-example}
\begin{lstlisting}
sequence_multiplicity test.fasta > out.fasta
\end{lstlisting}
Finds unique sequences in \emph{test.fasta} and writes them into
\emph{out.fasta}. Each sequence in the \emph{out.fasta} file contains
a suffix number telling you how many times this sequence occurred in
the original dataset. 
\lstset{language={},
  caption={Example a line in \emph{out.fasta}},
  label=lst-sequencemultiplicity-example-out}
\begin{lstlisting}
>sequence_2_2344 
\end{lstlisting}
Tells you that the third FASTA file, we count from 0 onwards in the
input file, appears 2344 times in the dataset.  

\subsection{Implementation}
Finding unique sequences is implemented in \emph{dataset.c} while
the interface is implemented in \emph{sequence\_multiplicity.c}. 
