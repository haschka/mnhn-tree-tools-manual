\section{find\_sequence\_in\_split\_sets}

\subsection{General}

The tool allows you to find a specific sequence among the clusters
obtained from an adaptive clustering run as performed, for instance, with the
tools outlined in section \ref{sec-adaptive-clust}.

\subsection{Usage}

The utility can be called like:
\lstset{language=bash,
  caption={Calling the \emph{find\_sequence\_in\_split\_sets} tool},
  label=lst-fseqsplit-call}
\begin{lstlisting}
find_sequence_in_split_set [fasta-ds] [fasta-seq] [split-sets]
\end{lstlisting}
with the arguments:
\begin{enumerate}
  \item \emph{fasta-ds} A FASTA file containing the entire dataset of
    sequences that an adaptive clustering run was performed on.
  \item \emph{fasta-seq} A FASTA file containing a singe sequence,
    that should be searched for in the obtained clusters
  \item \emph{split-sets} A list of binary files, that represent the
    different layers of clusters, obtained from an adaptive clustering
    run.
\end{enumerate}

\subsection{Example}
\lstset{language=bash,
  caption={Example of the \emph{find\_sequence\_in\_split\_sets} tool},
  label=lst-fseqsplit-example}
\begin{lstlisting}
find_sequence_in_split_set test.fasta seq.fasta /tmp/s*
\end{lstlisting}
The command shown above searches for the sequence defined
in \emph{seq.fasta}, which has to be part of the complete dataset
\emph{test.fasta},
in the different \emph{split-sets} selected with
\emph{/tmp/s*}. The wildcard here, as with most tools that work with
multiple \emph{split-sets}, will work as the files are numbered correctly for
such a selection during an adaptive clustering run.
The tool prints an output like the following:
\lstset{language={},
  caption={Output of the \emph{find\_sequence\_in\_split\_sets} tool},
  label=lst-fseqsplit-output}
\begin{lstlisting}
Sequence has internal ids: 
0 
Sequence is found in Clusters
L22C0 1 times
L23C0 1 times
L24C0 1 times
L25C0 1 times
L26C0 1 times
L27C0 1 times
\end{lstlisting}
The designation of 0 as internal id means that in FASTA outputs of
the clusters, such as those 
created with the tool \emph{split\_set\_to\_fasta} (c.f. section
\ref{sec-sstofasta}), the sequence is
named \emph{>sequence\_0} as it is,
in this case, the first sequence of the original dataset. The sequence
is found in clusters named \emph{LXCY}. Where $X$ represents the layer
number and $C$ the cluster number. As such the sequence is to be found in the
binary \emph{split-set} bearing the number $X$ at its end. Once this
binary \emph{split-set} is converted to clusters, i.e in FASTA form,
the sequence is found in the files with the suffix $Y$. One may also
say that in the tree built from such an adaptive run the sequence is
to be found in the tree layer $X$ in cluster $Y$.

\subsection{Implementation}
The interface as well as the tool itself is implemented in \newline
\emph{find\_sequence\_in\_split\_sets.c}.
