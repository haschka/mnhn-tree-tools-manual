\chapter{Installation}

\section{Generic Installation} \label{sec-inst-generic}

In this chapter we will provide you with the details that allow you to
install \emph{MNHN-Tree-Tools}. The primary distribution method is in the
form of source code on github: \newline
\url{https://gitlab.in2p3.fr/mnhn-tools/mnhn-tree-tools} \newline

For a full installation a machine with a UNIX or UNIX-like system
is required. The machine currently has to be of \emph{x86\_64}
architecture and support the \emph{AVX} instruction set. In order to
compile the full suite of tools you need a toolchain, consisting of:
\begin{itemize}
\item \emph{make:} Which is required to build the suite and various subsets
\item \emph{compiler/linker:} A working C compiler, the suite was found to
  compile with \emph{gcc} and \emph{clang}.
\item \emph{BLAS/LAPACK} An implementation of the BLAS/LAPACK
  libraries.
\item \emph{OpenCL:} A working implementation of the OpenCL
  language. The suite was tested to compile with the
  \emph{nvidia-opencl-toolkit} or the \emph{intel-opencl-sdk}.
\item \emph{SDL2:} An implementation of the SDL2 library.
\item \emph{MPI:} An implementation of the Message Passing Interface
  (MPI). The suite was tested with \emph{OpenMPI}.
\item \emph{PNG:} An implementation of the PNG graphics library.
\end{itemize}
You may make sure that you not only have the binary versions of the
supporting libraries installed but also the header files, otherwise
the software will not be able to compile. As such you may need to
install the developer \emph{-dev} versions of the libraries. In the
next step you may clone the \emph{MNHN-Tree-Tools} source code from github.
You can do this by issuing the following command in your UNIX
terminal:
\lstset{language=bash,
  caption={Cloning the source from github.},
  label=lst-git-clone}
\begin{lstlisting}                                                              
git clone https://gitlab.in2p3.fr/mnhn-tools/mnhn-tree-tools.git
cd mnhn-tree-tools
\end{lstlisting}
Cloned and entered the mnhn-tree-tools
directory you need to edit the head of the
make file to suite your UNIX systems needs, select the correct compiler
flags and linking commands to the libraries. You may consult the
forums and documentation of your System/Distribution to find the
correct values. In general \emph{CC} is the command of your compiler,
\emph{MPICC} the MPI wrapper of your compiler. \emph{CFLAGS} the
options that you want to pass to your compiler. The other options are
the flags required by the linker to find the appropriate libraries.
\lstset{language=make,
  caption={Head of the Makefile},
  label=lst-make-head}
\begin{lstlisting}                                                           
CC=gcc
MPICC=mpicc
CFLAGS= -O2 -march=native -ftree-vectorize -fomit-frame-pointer 
LAPACK=-llapack
MATH=-lm
PTHREAD=-pthread
OPENCL=-lOpenCL
PNG=-lpng
MPI=-lmpi
SDL=$(shell sdl2-config --cflags) $(shell sdl2-config --libs)
\end{lstlisting}
With your flags adjusted you may compile the code issuing
\lstset{language=bash,
  caption={Building all tools},
  label=lst-build}
\begin{lstlisting}
mkdir bin
make all
\end{lstlisting}
If you do not need the full functionality of \emph{MNHN-Tree-Tools} you can
install each tool individually by calling:
\lstset{language=bash,
  caption={Building individual tools},
  label=lst-make-head-individual}
\begin{lstlisting}
mkdir bin
make name-of-the-tool
\end{lstlisting}
This might especially be convenient if you are not able to install all
the libraries in question on your system as the tool that you might
really need might not depend on the library that you might not be able
to install.

\section{Quick Installation on Debian 10 / Ubuntu 20.04}
If you just want to get all tools working quickly, we provide you with the
following quick install script for \emph{Debian 10} and derived distributions:

\lstset{language=bash,
  caption={Installing MNHN-Tree-Tools on Debian 10},
  label=lst-debian-install}
\begin{lstlisting}
sudo apt-get install git build-essential libpng-dev libsdl2-dev \
 liblapack-dev libopenmpi-dev libpocl-dev ocl-icd-opencl-dev pocl-opencl-icd

git clone https://gitlab.in2p3.fr/mnhn-tools/mnhn-tree-tools.git

cd mnhn-tree-tools
mkdir bin
make all
cd bin

# make MNHN-TREE-TOOLS available from any folder ( this is temporary, you
# may modify your .bashrc and similar files to make this permanent
export PATH=$PATH:$PWD
\end{lstlisting}

For optimal performance we recommend you to use a more elaborate
installation as shown in section \ref{sec-inst-generic} as this allows
to run OpenCL code on the GPU and the use of optimized math
(Lapack) and MPI libraries that might be available on your system. 
