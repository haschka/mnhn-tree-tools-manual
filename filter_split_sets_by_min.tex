\section{filter\_split\_sets\_by\_min} \label{sec-filter-split}

\subsection{General}

The tools allows one to filter clusters obtained from an adaptive clustering
run, such as with the tools outlined in section \ref{sec-adaptive-clust}, that
contain less sequences than a certain treshold, and retain sequences that
contain more sequence than this treshold.

\subsection{Usage}

The tool is to be called like:

\lstset{language=bash,
  caption={Calling the \emph{filter\_split\_sets\_by\_min} tool},
  label=lst-filterssbymin-call}
\begin{lstlisting}
filter_split_sets_by_min [minimum] [filtered-split-sets-path-name] \
   [input-split-sets]
\end{lstlisting}
where the arguments are:
\begin{enumerate}
\item \emph{minimum} An interger representing the minimum number of
  sequences to be found in a cluster to be retained in the dataset
\item \emph{filtered-split-sets-path-name} A path and name where the filtered
  \emph{split-sets} should be stored.
\item \emph{input-split-sets} The unfiltered \emph{split-sets} to filter using
  this algorithm.
\end{enumerate}

\subsection{example}
\lstset{language=bash,
  caption={Example of the \emph{filter\_split\_sets\_by\_min} tool},
  label=lst-filterssbymin-example}
\begin{lstlisting}
filter_split_sets_by_min 100 /tmp/f100s /tmp/s*
\end{lstlisting}
Here the program takes the \emph{split-sets}, clusters of each layer,
and filters clusters that contain less then 100 sequences from the
dataset. The new dataset without these clusters is written to
\emph{/tmp/f100sX} where $X$ is the number for each layer in the
output splitset. The selection using the wildcard works as the
original split sets are name \emph{/tmp/sX} and the numbers $X$ are
created in a way by the adaptive clustering tools such that the
wildcard selection keeps them in correct order.

\subsection{implementation}
The code and the interface is implemented in
\emph{filter\_split\_set\_by\_min.c}. The filtering mechanism in
\emph{cluster\_io.c}
