\section{filter\_split\_sets\_by\_min} \label{sec-filter-split}

\subsection{General}

This tool allows one to filter clusters obtained from an adaptive clustering
run, as outlined in section \ref{sec-adaptive-clust}, that
contain less sequences than a certain threshold, and to retain clusters that
contain more sequences than this threshold.

\subsection{Usage}

The tool is to be called like:

\lstset{language=bash,
  caption={Calling the \emph{filter\_split\_sets\_by\_min} tool},
  label=lst-filterssbymin-call}
\begin{lstlisting}
filter_split_sets_by_min [minimum] [filtered-split-sets-path-name] \
   [input-split-sets]
\end{lstlisting}
where the arguments are:
\begin{enumerate}
\item \emph{minimum} An integer representing the minimum number of sequences
  a cluster has to contain in order to be retained in the dataset
\item \emph{filtered-split-sets-path-name} A path and name where the filtered
  \emph{split-sets} should be stored.
\item \emph{input-split-sets} The unfiltered \emph{split-sets} to filter using
  this algorithm.
\end{enumerate}

\subsection{Example}
\lstset{language=bash,
  caption={Example of the \emph{filter\_split\_sets\_by\_min} tool},
  label=lst-filterssbymin-example}
\begin{lstlisting}
filter_split_sets_by_min 100 /tmp/f100s /tmp/s*
\end{lstlisting}
Here, the program reads the \emph{split-sets}, clusters of each layer,
and filters clusters that contain less then 100 sequences from the
dataset. The new dataset without these clusters is written to
\emph{/tmp/f100sX} where $X$ is the number for each layer in the
output \emph{split-set}. The selection using the wildcard works as the
original \emph{split-set}s are named \emph{/tmp/sX} and the numbers $X$ are
created in a way by the adaptive clustering tools such that the
wildcard selection keeps them in correct order.

\subsection{implementation}
The code and the interface is implemented in \newline
\emph{filter\_split\_set\_by\_min.c}. The filtering mechanism in
\emph{cluster\_io.c}
