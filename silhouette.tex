\section{silhouette}

\subsection{General}

The \emph{silhouette} tool calculates the silhouette index comparing two
clusters or datasets stored in two FASTA files. The \emph{silhouette}
tool only works with nucleotide sequences.

\subsection{Usage}

One can call the \emph{silhouette} tool like:
\lstset{language=bash,
  caption={Calling the \emph{silhouette} tool},
  label=lst-silhouette-call}
\begin{lstlisting}
silhouette [fasta1] [fasta2] [n-threads]
\end{lstlisting}
with the following arguments:
\begin{enumerate}
\item \emph{fasta1}: A FASTA file containing a set of sequences.
\item \emph{fasta2}: An other FASTA file containing a set of sequences.
\item \emph{n-threads}: The number of threads that this computation might use.
\end{enumerate}

\subsection{Algorithm}

The tool calculates the silhouette index comparing the distances
internal to each dataset (distances between sequences within a FASTA
file), with the distances external to each dataset (distances between
sequences between two FASTA files).

The silhouette index is calculated using the following formula:

\begin{equation}
  S=\frac{1}{n_1}\sum_{i=1}^{n_{1}} s_{i},
\end{equation}
with $s_i$ defined to be:
\begin{equation}
  s_i=\left\{
  \begin{array}{l}
    a_i < b_i : 1-a_i/b_i \\
    a_i = b_i : 0 \\
    a_i > b_i : b_i/a_i-1
  \end{array} \right.
\end{equation}
and $a_i$ being the internal measure for distances to the $i$-th
sequence $A_i$ found in the first FASTA file \emph{fasta1} and hence,
\begin{equation}
  a_i = \frac{1}{n_a}\sum_{j=1}^{n_A}d_{\mathrm{SW}}(A_i,A_j).
\end{equation}
Likewise $b_i$ is the external measure for distances to the $i$-th
sequence $A_i$ in the first FASTA file and sequences $B$ found in the
second FASTA file. We write as such:
\begin{equation}
  b_i = \frac{1}{n_b}\sum_{j=1}^{n_B}d_{\mathrm{SW}}(A_i,B_j),
\end{equation}
with $d_{SW}(X,Y)$ being the Smith Waterman distance between two
sequences $X$ and $Y$ as defined in section
\ref{sec-dbscan-algorithm}. $n_A$ and $n_B$ are the number of
sequences found in the first $A$, and second $B$ FASTA file. 

\subsection{Example}
\lstset{language=bash,
  caption={Example of the \emph{silhouette} tool},
  label=lst-silhouette-example}
\begin{lstlisting}
silhouette test1.fasta test2.fasta 8
\end{lstlisting}
This example calculates the silhouette index between the dataset of
sequences found  in \emph{test1.fasta} and the dataset found in
\emph{test2.fasta}.

\subsection{Implementation}
The functions to calculate the silhouette index are implemented in
\emph{comparison.c}. The interface is implemented in
\emph{silhouette.c}.


