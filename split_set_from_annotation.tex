\section{split\_set\_from\_annotation} \label{sec-ssannotation}

\subsection{General}

This tool allows you to generate binary clustering sets such as those created
using by an adaptive clustering run by hand, annotating each cluster.

\subsection{Usage}
The tool can be used like:
\lstset{language=bash,
  caption={Calling the \emph{split\_set\_from\_annotation} tool},
  label=lst-ssannotation-call}
\begin{lstlisting}
split_set_from_annotation [fasta-ds] [annotation] [unique-annotations] \
  [split-set]
\end{lstlisting}
with the following arguments:
\begin{enumerate}
\item \emph{fasta-ds} A FASTA file that shall be clustered according to the
  annotations in the \emph{annotation} file.
\item \emph{annotation} The annotation file, containing a line for each
  sequence in the \emph{fasta-ds} file. An annotation can be any chosen word.
  If a line contains only a dash '-', the sequence will not be part of any of
  the possible clusters.
\item \emph{unique-annotations} The database of possible annotations.
  For each cluster the word used to annotate sequences in the \emph{annotation}
  file an entry containing the word corresponding to this cluster is made.
  A sole entry. The entries shall be made in cluster order that you wish
  to have in the resulting binary \emph{split-set} file.
\item \emph{split-set} The resulting binary clustering or \emph{split-set}
  file.
\end{enumerate}
\subsection{Example}
We imagine a fasta file:
\lstset{language={},
  caption={Example of a dataset clustered with the \emph{split\_set\_from\_annotation} tool},
  label=lst-ssannotation-example-fasta}
\begin{lstlisting}
>sequence1
AACCTT
>sequence2
AGCCTT
>sequence3
TTTTTT
>sequence4
AAACTT
>sequence5
AAACGT
>sequence6
AAAGCT
\end{lstlisting}
and we want to cluster this file into two clusters, one containing
sequence 1, and sequence 2 while the second cluster is composed of
sequence 4, 5 and 6. We would like to omit sequence 3 from the
clustering. As such we create an annotations file like:
\lstset{language={},
  caption={Example of an \emph{annotation} file for the \emph{split\_set\_from\_annotation} tool},
  label=lst-ssannotation-example-annotations}
\begin{lstlisting}
one
one
-
two
two
two
\end{lstlisting}
As shown we annotate all sequences of the first cluster with one, and
all sequences of the second cluster with two. We omit the third
sequence by annotating it with a dash '-'. Further we need to write a
file containing all the words that we used to annotate our clusters.
\lstset{language={},
  caption={Example of an \emph{unique-annotations} file for the \emph{split\_set\_from\_annotation} tool},
  label=lst-ssannotation-example-unique-annotations}
\begin{lstlisting}
one
two
\end{lstlisting}
With the files at hand we can call the tool:
\lstset{language={},
  caption={Example of the \emph{split\_set\_from\_annotation} tool},
  label=lst-ssannotation-example}
\begin{lstlisting}
split_set_from_annotation test.fasta annotation unique-annotations /tmp/out-set
\end{lstlisting}
and obtain the binary \emph{split-set} clustering for the clustering
at our wish in \emph{/tmp/out-set}.

\subsection{Implementation}
The tool is implemented in \emph{split\_set\_from\_annotation.c}.
