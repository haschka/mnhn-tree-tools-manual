\section{pca2densitymap} \label{sec-pca2densitymap}

\subsection{General}

The \emph{pca2densiymap} tool allows you to visualize the the sequence density
in a PCA file, as for instance created using the \emph{fasta2kmer} tool
outlined in section \ref{sec-fasta2kmer},
by generating a portable network graphics (PNG) image. This basically 
produces a histogram by laying a rectangular grid across a projection
into two/three dimensional subspace, spun by principal components. The
tool then counts how many sequences are found within a square/cube
and attributes this value to a pixel/voxel in the final
image/images. The 3D version creates a stack of images. To overcome
the limitations of an image based tool such as this one,
one may refer to the \emph{pca2densityfile} tool outlined in section
\ref{sec-pca2densityfile}. 

\subsection{Usage}

The tool can be run using the following commands:
\lstset{language=bash,
  caption={Calling the \emph{pca2densitymap} tool},
  label=lst-pca2densitymap-call}
\begin{lstlisting}
pca2densitymap [fasta] [pca] [dimensions] [2D or 3D] \
  [invertal-length] [png-file-or-files]
\end{lstlisting}
with the arguments:
\begin{enumerate}
  \item \emph{fasta} The original FASTA file that has been transformed
    into a k-mer representation which was projected into a subspace
    spun by principal components. For instance by using the tools
    \emph{fasta2kmer} and \emph{kmer2pca} as described in sections
    \ref{sec-fasta2kmer} and \ref{sec-kmer2pca}.
  \item \emph{pca} A file containing the projections onto the
    principal components.
  \item \emph{dimensions} An integer describing how many dimensions
    the PCA subspace consists of.
  \item \emph{2D or 3D} One may choose 2 for a 2-dimensional output or 3 for a
    3-dimensional output. 3-dimensional outputs are not fully tested
    though.
  \item \emph{interval-length} The grid spacing for our histogram.
  \item \emph{png-file-or-files} A path for the resulting PNG file. In
    case of a 3D output a whole stack of images with individual
    numbers for each slice will be created.  
\end{enumerate}

\subsection{Example}
\lstset{language=bash,
  caption={Example of the \emph{pca2densitymap} tool},
  label=lst-pca2densitymap-example}
\begin{lstlisting}
pca2densitymap test.fasta test.pca 7 2 0.1 /temp/out.png
\end{lstlisting}
In this example an image, \emph{out.png} will be created that contains
pixels ranging from
white, representing the highest number of sequences found in a  grid
tile, to black, with no sequences in the boundaries of a grid
tile. As the image can only encode 8 bits per pixel only 255 different
gray levels are available. This might cause, due to the rescaling,
that only a few white points exist in the image if the dataset
contains very large peaks. \emph{test.fasta} is the sequence
dataset, \emph{test.pca} its projection into a 7 dimensional PCA
subspace. The tool always chooses to only treat either the first two or
three dimensions, for the 2D or 3D case respectively.
Further in this example, we ask for a two dimensional output
with a grid spacing of
0.1 [k-mer frequencies]. The tool yields some information about
the image generated:
\lstset{language={},
  caption={Text output of the \emph{pca2densitymap} tool},
  label=lst-pca2densitymap-txt-out}
\begin{lstlisting}
   shift x:  8.45373440
   shift y:  5.56738997
n points x:    122
n points y:     89
factor 255/maximum(intensity): 0.10814250
\end{lstlisting}
where shift x and y are origin of the image, in general on the upper
left. n Points x and y are the number of pixels according to the
spacing chosen that are calculated and stored in the output PNG
file. The last value is the scalefactor that has been used in order to
scale the densities from 0 to 255. A sample
of an image generated using this tool is highlighted in figure
\ref{fig-pca2densitymap}.
\begin{figure}
  \begin{center}
    \includegraphics{pca-density.png}
    \caption{The sequence density in a two dimensional PCA subspace of
    a k-mer representation. The image was upscaled and color inverted
    for printing using the Gnu Image Manipulation Program (GIMP)
    \cite{gimp}.}
    \label{fig-pca2densitymap}
  \end{center}
\end{figure}

\subsection{Implementation}
The interface to this tool is implemented in \emph{pca2densitymap.c}.
The engine of the tool resides in \emph{density.c}.
