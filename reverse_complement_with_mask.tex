\section{reverse\_complement\_with\_mask} \label{sec-revcomp}

\subsection{General}

This tool allows you to reverse complement selected sequences in a FASTA file.
A mask text file has to be provided to indicate which sequences shall
be reversed. This tool not only changes read direction of the
selected sequences but also correctly complements the nucleotides.

\subsection{Usage}

The \emph{reverse\_complement\_with\_mask} tool can be called like:
\lstset{language=bash,
  caption={Calling the \emph{reverse\_complement\_with\_mask\_tool} tool},
  label=lst-revcomp-call}
\begin{lstlisting}
reverse_with_mask [fasta] [mask] > [fasta-out]
\end{lstlisting}
with the following arguments:
\begin{enumerate}
  \item \emph{fasta} A FASTA file to reverse complement sequences in.
  \item \emph{mask} A mask file indicating the utility which sequences to
    reverse complement. The file has to contain a line with a number for each
    sequence. If the number matches 1 on line of the mask file, the
    sequence corresponding to this line is replaced by its reverse
    complement in the resulting FASTA file.
  \item \emph{fasta-out} The FASTA dataset containing all sequences
    together with the reverse complemented at location where the mask matched.
\end{enumerate}

\subsection{Example}
We imagine the following FASTA file with three sequences:
\lstset{language={},
  caption={Example of the \emph{reverse\_complement\_with\_mask} tool -
    FASTA file},
  label=lst-revcomp-fasta}
\begin{lstlisting}
> sequence_1
ACTG
> sequence_2
ACCT
> sequence_3
GGAT
\end{lstlisting}
As we would like to reverse complement the second sequence, we write
a mask file with the following contents:
\lstset{language={},
  caption={Example of the \emph{reverse\_complement\_with\_mask} tool -
    Mask file},
  label=lst-revcomp-mask}
\begin{lstlisting}
0
1
0
\end{lstlisting}
Running the tool by issuing the following command:
\lstset{language=bash,
  caption={Example of the \emph{reverse\_complement\_with\_mask} tool},
  label=lst-revcomp-example}
\begin{lstlisting}
reverse_complement_with_mask test.fasta test.mask > out.fasta
\end{lstlisting}
produces the following \emph{out.fasta} file:
\lstset{language={},
  caption={Example of the \emph{reverse\_complement\_with\_mask} tool -
    Output FASTA file},
  label=lst-revcomp-fastaout}
\begin{lstlisting}
> sequence_1
ACTG
> sequence_2
AGGT
> sequence_3
GGAT
\end{lstlisting}

\subsection{Implementation}
The tools interface is implemented in \emph{reverse\_complement\_with\_mask.c}.
The code that reverses the sequences resides in \emph{dataset.c}.
