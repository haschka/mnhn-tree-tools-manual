\section{replace\_N\_sequence}


\subsection{General}

\emph{replace\_N\_sequence} Is a tool to replace sequences that are
holding N letters in them with all possible combinations
possible. This tool currently only works with nucleotide sequences.

\subsection{Usage}

The tool can be called like:
\lstset{language=bash,
  caption={Calling the \emph{replace\_N\_sequence} tool},
  label=lst-repnseq-call}
\begin{lstlisting}
replace_N_sequence [fasta] [sequence-index] [fasta-out]
\end{lstlisting}
where the arguments are as follows:
\begin{enumerate}
\item \emph{fasta} A FASTA file holding sequences
\item \emph{sequence-index} An index to select a sequence from the
\item \emph{fasta} file. In this tool the first sequence is selected by
  1.
\item \emph{fasta-out} All possible permutations by replacing the N
  letters in the sequence.
\end{enumerate}

\subsection{Example}
Let us imagine a fasta file like the following:
\lstset{language={},
  caption={Fasta example for the \emph{replace\_N\_sequence} tool},
  label=lst-repnseq-fastaex}
\begin{lstlisting}
>sequence1
ACGT
>sequence2
ANNT
>sequence3
ANTG
\end{lstlisting}
We would like to get all sequences possible by replacing the N letters
in the second sequence, hence we call the utility like:
\lstset{language=bash,
  caption={Fasta example for the \emph{find\_satellite} tool},
  label=lst-repnseq-example}
\begin{lstlisting}
replace_N_sequence test.fasta 2 /tmp/out.fasta
\end{lstlisting}
And get a \emph{/tmp/out.fasta} that looks like:
\lstset{language={},
  caption={Fasta example for the \emph{replace\_N\_sequence} tool},
  label=lst-repnseq-fastaout}
\begin{lstlisting}
>sequence_0
AAAT
>sequence_1
ACAT
>sequence_2
AGAT
>sequence_3
ATAT
>sequence_4
AACT
>sequence_5
ACCT
>sequence_6
AGCT
>sequence_7
ATCT
>sequence_8
AAGT
>sequence_9
ACGT
>sequence_10
AGGT
>sequence_11
ATGT
>sequence_12
AATT
>sequence_13
ACTT
>sequence_14
AGTT
>sequence_15
ATTT
\end{lstlisting}

\subsection{implementation}
The tools interface is implemented in
\emph{replace\_N\_sequence.c}. The mechanism is implemented in
\emph{kmers.c}.


