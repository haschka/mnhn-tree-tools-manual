\section{split\_set\_to\_fasta} \label{sec-sstofasta}

This tool creates FASTA files for clusters found in a \emph{split-set}
stored in binary form like the ones generated by the
\emph{adaptive\_clustering\_X} tools, highlighted in section
\ref{sec-adaptive-clust}.

\subsection{Usage}
The \emph{split\_set\_to\_fasta} tool can be called in the following way:
\lstset{language=bash,
  caption={Calling the \emph{split\_set\_to\_fasta} tool},
  label=lst-sstofasta-call}
\begin{lstlisting}
  split_set_to_fasta [fasta] [split-set] [out-path-name]
\end{lstlisting}
with the arguments:
\begin{enumerate}
  \item \emph{fasta} A FASTA file containing the whole dataset from
    that is partitioned by a \emph{split-set}.
  \item \emph{split-set} The binary \emph{split-set} containing the
    clusters.
  \item \emph{out-path-name} The path and name of the resulting FASTA
    files that are suffixed with unique number for each cluster in
    stored in the \emph{split-set}
\end{enumerate}

\subsection{example}
\lstset{language=bash,
  caption={Example of the \emph{split\_set\_to\_fasta} tool},
  label=lst-sstofasta-example}
\begin{lstlisting}
split_set_to_fasta test.fasta /tmp/s0023 /tmp/fa0023-
\end{lstlisting}
Here the \emph{split-set} \emph{/tmp/s0023} is written to individual
FASTA files per clusters stored to
\emph{/tmp/fa0023-X} where X is a number for each individual cluster.
\emph{test.fasta} holds the original dataset
which is needed as the \emph{split-set} file only stores indices and
not the sequences itself.

\subsection{Implementation}
The tool is implemented in \emph{split\_set\_to\_fasta.c}. 
